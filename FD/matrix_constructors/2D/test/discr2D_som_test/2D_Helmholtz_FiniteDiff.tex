
% Default to the notebook output style

    


% Inherit from the specified cell style.




    
\documentclass[a4paper, landscape, 11pt]{article}

    
    
    \usepackage[T1]{fontenc}
    % Nicer default font than Computer Modern for most use cases
    \usepackage{palatino}

    % Basic figure setup, for now with no caption control since it's done
    % automatically by Pandoc (which extracts ![](path) syntax from Markdown).
    \usepackage{graphicx}
    % We will generate all images so they have a width \maxwidth. This means
    % that they will get their normal width if they fit onto the page, but
    % are scaled down if they would overflow the margins.
    \makeatletter
    \def\maxwidth{\ifdim\Gin@nat@width>\linewidth\linewidth
    \else\Gin@nat@width\fi}
    \makeatother
    \let\Oldincludegraphics\includegraphics
    % Set max figure width to be 80% of text width, for now hardcoded.
    \renewcommand{\includegraphics}[1]{\Oldincludegraphics[width=.8\maxwidth]{#1}}
    % Ensure that by default, figures have no caption (until we provide a
    % proper Figure object with a Caption API and a way to capture that
    % in the conversion process - todo).
    \usepackage{caption}
    \DeclareCaptionLabelFormat{nolabel}{}
    \captionsetup{labelformat=nolabel}

    \usepackage{adjustbox} % Used to constrain images to a maximum size 
    \usepackage{xcolor} % Allow colors to be defined
    \usepackage{enumerate} % Needed for markdown enumerations to work
    \usepackage{geometry} % Used to adjust the document margins
    \usepackage{amsmath} % Equations
    \usepackage{amssymb} % Equations
    \usepackage{textcomp} % defines textquotesingle
    % Hack from http://tex.stackexchange.com/a/47451/13684:
    \AtBeginDocument{%
        \def\PYZsq{\textquotesingle}% Upright quotes in Pygmentized code
    }
    \usepackage{upquote} % Upright quotes for verbatim code
    \usepackage{eurosym} % defines \euro
    \usepackage[mathletters]{ucs} % Extended unicode (utf-8) support
    \usepackage[utf8x]{inputenc} % Allow utf-8 characters in the tex document
    \usepackage{fancyvrb} % verbatim replacement that allows latex
    \usepackage{grffile} % extends the file name processing of package graphics 
                         % to support a larger range 
    % The hyperref package gives us a pdf with properly built
    % internal navigation ('pdf bookmarks' for the table of contents,
    % internal cross-reference links, web links for URLs, etc.)
    \usepackage{hyperref}
    \usepackage{longtable} % longtable support required by pandoc >1.10
    \usepackage{booktabs}  % table support for pandoc > 1.12.2
    \usepackage[normalem]{ulem} % ulem is needed to support strikethroughs (\sout)
                                % normalem makes italics be italics, not underlines
    

    
    
    % Colors for the hyperref package
    \definecolor{urlcolor}{rgb}{0,.145,.698}
    \definecolor{linkcolor}{rgb}{.71,0.21,0.01}
    \definecolor{citecolor}{rgb}{.12,.54,.11}

    % ANSI colors
    \definecolor{ansi-black}{HTML}{3E424D}
    \definecolor{ansi-black-intense}{HTML}{282C36}
    \definecolor{ansi-red}{HTML}{E75C58}
    \definecolor{ansi-red-intense}{HTML}{B22B31}
    \definecolor{ansi-green}{HTML}{00A250}
    \definecolor{ansi-green-intense}{HTML}{007427}
    \definecolor{ansi-yellow}{HTML}{DDB62B}
    \definecolor{ansi-yellow-intense}{HTML}{B27D12}
    \definecolor{ansi-blue}{HTML}{208FFB}
    \definecolor{ansi-blue-intense}{HTML}{0065CA}
    \definecolor{ansi-magenta}{HTML}{D160C4}
    \definecolor{ansi-magenta-intense}{HTML}{A03196}
    \definecolor{ansi-cyan}{HTML}{60C6C8}
    \definecolor{ansi-cyan-intense}{HTML}{258F8F}
    \definecolor{ansi-white}{HTML}{C5C1B4}
    \definecolor{ansi-white-intense}{HTML}{A1A6B2}

    % commands and environments needed by pandoc snippets
    % extracted from the output of `pandoc -s`
    \providecommand{\tightlist}{%
      \setlength{\itemsep}{0pt}\setlength{\parskip}{0pt}}
    \DefineVerbatimEnvironment{Highlighting}{Verbatim}{commandchars=\\\{\}}
    % Add ',fontsize=\small' for more characters per line
    \newenvironment{Shaded}{}{}
    \newcommand{\KeywordTok}[1]{\textcolor[rgb]{0.00,0.44,0.13}{\textbf{{#1}}}}
    \newcommand{\DataTypeTok}[1]{\textcolor[rgb]{0.56,0.13,0.00}{{#1}}}
    \newcommand{\DecValTok}[1]{\textcolor[rgb]{0.25,0.63,0.44}{{#1}}}
    \newcommand{\BaseNTok}[1]{\textcolor[rgb]{0.25,0.63,0.44}{{#1}}}
    \newcommand{\FloatTok}[1]{\textcolor[rgb]{0.25,0.63,0.44}{{#1}}}
    \newcommand{\CharTok}[1]{\textcolor[rgb]{0.25,0.44,0.63}{{#1}}}
    \newcommand{\StringTok}[1]{\textcolor[rgb]{0.25,0.44,0.63}{{#1}}}
    \newcommand{\CommentTok}[1]{\textcolor[rgb]{0.38,0.63,0.69}{\textit{{#1}}}}
    \newcommand{\OtherTok}[1]{\textcolor[rgb]{0.00,0.44,0.13}{{#1}}}
    \newcommand{\AlertTok}[1]{\textcolor[rgb]{1.00,0.00,0.00}{\textbf{{#1}}}}
    \newcommand{\FunctionTok}[1]{\textcolor[rgb]{0.02,0.16,0.49}{{#1}}}
    \newcommand{\RegionMarkerTok}[1]{{#1}}
    \newcommand{\ErrorTok}[1]{\textcolor[rgb]{1.00,0.00,0.00}{\textbf{{#1}}}}
    \newcommand{\NormalTok}[1]{{#1}}
    
    % Additional commands for more recent versions of Pandoc
    \newcommand{\ConstantTok}[1]{\textcolor[rgb]{0.53,0.00,0.00}{{#1}}}
    \newcommand{\SpecialCharTok}[1]{\textcolor[rgb]{0.25,0.44,0.63}{{#1}}}
    \newcommand{\VerbatimStringTok}[1]{\textcolor[rgb]{0.25,0.44,0.63}{{#1}}}
    \newcommand{\SpecialStringTok}[1]{\textcolor[rgb]{0.73,0.40,0.53}{{#1}}}
    \newcommand{\ImportTok}[1]{{#1}}
    \newcommand{\DocumentationTok}[1]{\textcolor[rgb]{0.73,0.13,0.13}{\textit{{#1}}}}
    \newcommand{\AnnotationTok}[1]{\textcolor[rgb]{0.38,0.63,0.69}{\textbf{\textit{{#1}}}}}
    \newcommand{\CommentVarTok}[1]{\textcolor[rgb]{0.38,0.63,0.69}{\textbf{\textit{{#1}}}}}
    \newcommand{\VariableTok}[1]{\textcolor[rgb]{0.10,0.09,0.49}{{#1}}}
    \newcommand{\ControlFlowTok}[1]{\textcolor[rgb]{0.00,0.44,0.13}{\textbf{{#1}}}}
    \newcommand{\OperatorTok}[1]{\textcolor[rgb]{0.40,0.40,0.40}{{#1}}}
    \newcommand{\BuiltInTok}[1]{{#1}}
    \newcommand{\ExtensionTok}[1]{{#1}}
    \newcommand{\PreprocessorTok}[1]{\textcolor[rgb]{0.74,0.48,0.00}{{#1}}}
    \newcommand{\AttributeTok}[1]{\textcolor[rgb]{0.49,0.56,0.16}{{#1}}}
    \newcommand{\InformationTok}[1]{\textcolor[rgb]{0.38,0.63,0.69}{\textbf{\textit{{#1}}}}}
    \newcommand{\WarningTok}[1]{\textcolor[rgb]{0.38,0.63,0.69}{\textbf{\textit{{#1}}}}}
    
    
    % Define a nice break command that doesn't care if a line doesn't already
    % exist.
    \def\br{\hspace*{\fill} \\* }
    % Math Jax compatability definitions
    \def\gt{>}
    \def\lt{<}
    % Document parameters
    \title{Discretization of the Helmholtz equation in 2D with finite differences}
    
    
    

    % Pygments definitions
    
\makeatletter
\def\PY@reset{\let\PY@it=\relax \let\PY@bf=\relax%
    \let\PY@ul=\relax \let\PY@tc=\relax%
    \let\PY@bc=\relax \let\PY@ff=\relax}
\def\PY@tok#1{\csname PY@tok@#1\endcsname}
\def\PY@toks#1+{\ifx\relax#1\empty\else%
    \PY@tok{#1}\expandafter\PY@toks\fi}
\def\PY@do#1{\PY@bc{\PY@tc{\PY@ul{%
    \PY@it{\PY@bf{\PY@ff{#1}}}}}}}
\def\PY#1#2{\PY@reset\PY@toks#1+\relax+\PY@do{#2}}

\expandafter\def\csname PY@tok@il\endcsname{\def\PY@tc##1{\textcolor[rgb]{0.40,0.40,0.40}{##1}}}
\expandafter\def\csname PY@tok@k\endcsname{\let\PY@bf=\textbf\def\PY@tc##1{\textcolor[rgb]{0.00,0.50,0.00}{##1}}}
\expandafter\def\csname PY@tok@ch\endcsname{\let\PY@it=\textit\def\PY@tc##1{\textcolor[rgb]{0.25,0.50,0.50}{##1}}}
\expandafter\def\csname PY@tok@gd\endcsname{\def\PY@tc##1{\textcolor[rgb]{0.63,0.00,0.00}{##1}}}
\expandafter\def\csname PY@tok@gt\endcsname{\def\PY@tc##1{\textcolor[rgb]{0.00,0.27,0.87}{##1}}}
\expandafter\def\csname PY@tok@vc\endcsname{\def\PY@tc##1{\textcolor[rgb]{0.10,0.09,0.49}{##1}}}
\expandafter\def\csname PY@tok@c\endcsname{\let\PY@it=\textit\def\PY@tc##1{\textcolor[rgb]{0.25,0.50,0.50}{##1}}}
\expandafter\def\csname PY@tok@ne\endcsname{\let\PY@bf=\textbf\def\PY@tc##1{\textcolor[rgb]{0.82,0.25,0.23}{##1}}}
\expandafter\def\csname PY@tok@sh\endcsname{\def\PY@tc##1{\textcolor[rgb]{0.73,0.13,0.13}{##1}}}
\expandafter\def\csname PY@tok@gu\endcsname{\let\PY@bf=\textbf\def\PY@tc##1{\textcolor[rgb]{0.50,0.00,0.50}{##1}}}
\expandafter\def\csname PY@tok@gi\endcsname{\def\PY@tc##1{\textcolor[rgb]{0.00,0.63,0.00}{##1}}}
\expandafter\def\csname PY@tok@s\endcsname{\def\PY@tc##1{\textcolor[rgb]{0.73,0.13,0.13}{##1}}}
\expandafter\def\csname PY@tok@cp\endcsname{\def\PY@tc##1{\textcolor[rgb]{0.74,0.48,0.00}{##1}}}
\expandafter\def\csname PY@tok@s2\endcsname{\def\PY@tc##1{\textcolor[rgb]{0.73,0.13,0.13}{##1}}}
\expandafter\def\csname PY@tok@sr\endcsname{\def\PY@tc##1{\textcolor[rgb]{0.73,0.40,0.53}{##1}}}
\expandafter\def\csname PY@tok@kp\endcsname{\def\PY@tc##1{\textcolor[rgb]{0.00,0.50,0.00}{##1}}}
\expandafter\def\csname PY@tok@mi\endcsname{\def\PY@tc##1{\textcolor[rgb]{0.40,0.40,0.40}{##1}}}
\expandafter\def\csname PY@tok@err\endcsname{\def\PY@bc##1{\setlength{\fboxsep}{0pt}\fcolorbox[rgb]{1.00,0.00,0.00}{1,1,1}{\strut ##1}}}
\expandafter\def\csname PY@tok@gs\endcsname{\let\PY@bf=\textbf}
\expandafter\def\csname PY@tok@nd\endcsname{\def\PY@tc##1{\textcolor[rgb]{0.67,0.13,1.00}{##1}}}
\expandafter\def\csname PY@tok@vg\endcsname{\def\PY@tc##1{\textcolor[rgb]{0.10,0.09,0.49}{##1}}}
\expandafter\def\csname PY@tok@se\endcsname{\let\PY@bf=\textbf\def\PY@tc##1{\textcolor[rgb]{0.73,0.40,0.13}{##1}}}
\expandafter\def\csname PY@tok@nl\endcsname{\def\PY@tc##1{\textcolor[rgb]{0.63,0.63,0.00}{##1}}}
\expandafter\def\csname PY@tok@mh\endcsname{\def\PY@tc##1{\textcolor[rgb]{0.40,0.40,0.40}{##1}}}
\expandafter\def\csname PY@tok@gr\endcsname{\def\PY@tc##1{\textcolor[rgb]{1.00,0.00,0.00}{##1}}}
\expandafter\def\csname PY@tok@si\endcsname{\let\PY@bf=\textbf\def\PY@tc##1{\textcolor[rgb]{0.73,0.40,0.53}{##1}}}
\expandafter\def\csname PY@tok@w\endcsname{\def\PY@tc##1{\textcolor[rgb]{0.73,0.73,0.73}{##1}}}
\expandafter\def\csname PY@tok@sb\endcsname{\def\PY@tc##1{\textcolor[rgb]{0.73,0.13,0.13}{##1}}}
\expandafter\def\csname PY@tok@c1\endcsname{\let\PY@it=\textit\def\PY@tc##1{\textcolor[rgb]{0.25,0.50,0.50}{##1}}}
\expandafter\def\csname PY@tok@go\endcsname{\def\PY@tc##1{\textcolor[rgb]{0.53,0.53,0.53}{##1}}}
\expandafter\def\csname PY@tok@cpf\endcsname{\let\PY@it=\textit\def\PY@tc##1{\textcolor[rgb]{0.25,0.50,0.50}{##1}}}
\expandafter\def\csname PY@tok@nv\endcsname{\def\PY@tc##1{\textcolor[rgb]{0.10,0.09,0.49}{##1}}}
\expandafter\def\csname PY@tok@kn\endcsname{\let\PY@bf=\textbf\def\PY@tc##1{\textcolor[rgb]{0.00,0.50,0.00}{##1}}}
\expandafter\def\csname PY@tok@mf\endcsname{\def\PY@tc##1{\textcolor[rgb]{0.40,0.40,0.40}{##1}}}
\expandafter\def\csname PY@tok@sd\endcsname{\let\PY@it=\textit\def\PY@tc##1{\textcolor[rgb]{0.73,0.13,0.13}{##1}}}
\expandafter\def\csname PY@tok@kr\endcsname{\let\PY@bf=\textbf\def\PY@tc##1{\textcolor[rgb]{0.00,0.50,0.00}{##1}}}
\expandafter\def\csname PY@tok@vi\endcsname{\def\PY@tc##1{\textcolor[rgb]{0.10,0.09,0.49}{##1}}}
\expandafter\def\csname PY@tok@o\endcsname{\def\PY@tc##1{\textcolor[rgb]{0.40,0.40,0.40}{##1}}}
\expandafter\def\csname PY@tok@no\endcsname{\def\PY@tc##1{\textcolor[rgb]{0.53,0.00,0.00}{##1}}}
\expandafter\def\csname PY@tok@ni\endcsname{\let\PY@bf=\textbf\def\PY@tc##1{\textcolor[rgb]{0.60,0.60,0.60}{##1}}}
\expandafter\def\csname PY@tok@nc\endcsname{\let\PY@bf=\textbf\def\PY@tc##1{\textcolor[rgb]{0.00,0.00,1.00}{##1}}}
\expandafter\def\csname PY@tok@nf\endcsname{\def\PY@tc##1{\textcolor[rgb]{0.00,0.00,1.00}{##1}}}
\expandafter\def\csname PY@tok@nb\endcsname{\def\PY@tc##1{\textcolor[rgb]{0.00,0.50,0.00}{##1}}}
\expandafter\def\csname PY@tok@cm\endcsname{\let\PY@it=\textit\def\PY@tc##1{\textcolor[rgb]{0.25,0.50,0.50}{##1}}}
\expandafter\def\csname PY@tok@kd\endcsname{\let\PY@bf=\textbf\def\PY@tc##1{\textcolor[rgb]{0.00,0.50,0.00}{##1}}}
\expandafter\def\csname PY@tok@bp\endcsname{\def\PY@tc##1{\textcolor[rgb]{0.00,0.50,0.00}{##1}}}
\expandafter\def\csname PY@tok@sx\endcsname{\def\PY@tc##1{\textcolor[rgb]{0.00,0.50,0.00}{##1}}}
\expandafter\def\csname PY@tok@na\endcsname{\def\PY@tc##1{\textcolor[rgb]{0.49,0.56,0.16}{##1}}}
\expandafter\def\csname PY@tok@mo\endcsname{\def\PY@tc##1{\textcolor[rgb]{0.40,0.40,0.40}{##1}}}
\expandafter\def\csname PY@tok@mb\endcsname{\def\PY@tc##1{\textcolor[rgb]{0.40,0.40,0.40}{##1}}}
\expandafter\def\csname PY@tok@sc\endcsname{\def\PY@tc##1{\textcolor[rgb]{0.73,0.13,0.13}{##1}}}
\expandafter\def\csname PY@tok@m\endcsname{\def\PY@tc##1{\textcolor[rgb]{0.40,0.40,0.40}{##1}}}
\expandafter\def\csname PY@tok@s1\endcsname{\def\PY@tc##1{\textcolor[rgb]{0.73,0.13,0.13}{##1}}}
\expandafter\def\csname PY@tok@gh\endcsname{\let\PY@bf=\textbf\def\PY@tc##1{\textcolor[rgb]{0.00,0.00,0.50}{##1}}}
\expandafter\def\csname PY@tok@kc\endcsname{\let\PY@bf=\textbf\def\PY@tc##1{\textcolor[rgb]{0.00,0.50,0.00}{##1}}}
\expandafter\def\csname PY@tok@ge\endcsname{\let\PY@it=\textit}
\expandafter\def\csname PY@tok@nn\endcsname{\let\PY@bf=\textbf\def\PY@tc##1{\textcolor[rgb]{0.00,0.00,1.00}{##1}}}
\expandafter\def\csname PY@tok@ow\endcsname{\let\PY@bf=\textbf\def\PY@tc##1{\textcolor[rgb]{0.67,0.13,1.00}{##1}}}
\expandafter\def\csname PY@tok@nt\endcsname{\let\PY@bf=\textbf\def\PY@tc##1{\textcolor[rgb]{0.00,0.50,0.00}{##1}}}
\expandafter\def\csname PY@tok@ss\endcsname{\def\PY@tc##1{\textcolor[rgb]{0.10,0.09,0.49}{##1}}}
\expandafter\def\csname PY@tok@cs\endcsname{\let\PY@it=\textit\def\PY@tc##1{\textcolor[rgb]{0.25,0.50,0.50}{##1}}}
\expandafter\def\csname PY@tok@gp\endcsname{\let\PY@bf=\textbf\def\PY@tc##1{\textcolor[rgb]{0.00,0.00,0.50}{##1}}}
\expandafter\def\csname PY@tok@kt\endcsname{\def\PY@tc##1{\textcolor[rgb]{0.69,0.00,0.25}{##1}}}

\def\PYZbs{\char`\\}
\def\PYZus{\char`\_}
\def\PYZob{\char`\{}
\def\PYZcb{\char`\}}
\def\PYZca{\char`\^}
\def\PYZam{\char`\&}
\def\PYZlt{\char`\<}
\def\PYZgt{\char`\>}
\def\PYZsh{\char`\#}
\def\PYZpc{\char`\%}
\def\PYZdl{\char`\$}
\def\PYZhy{\char`\-}
\def\PYZsq{\char`\'}
\def\PYZdq{\char`\"}
\def\PYZti{\char`\~}
% for compatibility with earlier versions
\def\PYZat{@}
\def\PYZlb{[}
\def\PYZrb{]}
\makeatother


    % Exact colors from NB
    \definecolor{incolor}{rgb}{0.0, 0.0, 0.5}
    \definecolor{outcolor}{rgb}{0.545, 0.0, 0.0}



    
    % Prevent overflowing lines due to hard-to-break entities
    \sloppy 
    % Setup hyperref package
    \hypersetup{
      breaklinks=true,  % so long urls are correctly broken across lines
      colorlinks=true,
      urlcolor=urlcolor,
      linkcolor=linkcolor,
      citecolor=citecolor,
      }
    % Slightly bigger margins than the latex defaults
    
    \geometry{verbose,tmargin=1in,bmargin=1in,lmargin=1in,rmargin=1in}
    
    

    \begin{document}
    
    
    \maketitle
  

    In this notebook we compute the stencils for the finite difference
discretization of the Helmholtz boundary value problem with Sommerfeld
boundary conditions in \(\Omega=(0,1)^2\):

\begin{align*}-\Delta u - k^2 u &= f \text{ in $\Omega$}\\
 \partial_{n}u - iku &=g \text{ on $\partial \Omega$}
\end{align*}

Let \(n_x,n_y\) be the number of interior points in the \(x\) and \(y\)
directions (including the endpoints), \(h_x=1/(n_x+1), h_y=1/(n_y+1)\)
and \(G=\{(x_i,y_j): 0 \leq i \leq n_x+1, 1 \leq j \leq n_y+1\}\) be the
corresponding grid. We first compute the equations for the interior
points.
    For the point \(x_{ij}=(ih_x,jh_y)\), we obtain the following equation
\[f_{ij}h_{x}^{2}h_{y}^{2}=- h_{x}^{2} \left(u_{{(i,j+1)}} + u_{{(i,j-1)}}\right) - h_{y}^{2} \left(u_{{(i+1,j)}} + u_{{(i-1,j)}}\right) + u_{{(i,j)}} \left(- h_{x}^{2} h_{y}^{2} k_{ij}^{2} + 2 h_{x}^{2} + 2 h_{y}^{2}\right),\]
equivalent to,
\[ f_{ij}=- h_{y}^{-2} \left(u_{{(i,j+1)}} + u_{{(i,j-1)}}\right) - h_{x}^{-2} \left(u_{{(i+1,j)}} + u_{{(i-1,j)}}\right) + u_{{(i,j)}} \left(-k_{ij}^{2} + 2 h_{y}^{-2} + 2 h_{x}^{-2}\right)\]

    We set now the linear equation for the non-corner points on the
boundary. We begin with points of the form \(x_{(0,j)}=(0,jh_y)\) on the
west boundary \((x=0, 1<y<1)\). At the point \(x_{(0,j)}\) the boundary
condition equals \(-u_{x(0,j)}-ik_{(0,j)}u_{(0,j)}=g_{(0,j)}\). We
approximate the derivative with forward differences and obtain
\[u_{x(0,j)}=h_x^{-1}(u_{(1,j)}-u_{(0,j)})-(1/2)h_xu_{xx(0,j)}+O(h_x^2)\]

    The second derivative \(u_{xx(0,j)}\) in the previous expression can be
substituted using the equation \(-u_{xx}-u_{yy}-k^2u=f\) (extended by
continuity to the boundary point)
    
    \[- f_{(0,j)} - k_{(0,j)}^{2} u_{(0,j)} - u_{yy(0,j)}\]


    And the second derivative \(u_{yy(0\,j)}\) can be approximated using
central differences (with order \(O(h_y^2)\)) to obtain
    
    \[f_{(0,j)} + k_{(0,j)}^{2} u_{(0,j)} + \frac{1}{h_{y}^{2}} \left(- 2 u_{(0,j)} + u_{(0,j+1)} + u_{(0,j-1)}\right)\]

    We substitute the expression for the second derivative in the expression
of the approximation of the first derivative, to obtain

    \[- \frac{h_{x}}{2} \left(- f_{(0,j)} - k_{(0,j)}^{2} u_{(0,j)} - \frac{1}{h_{y}^{2}} \left(- 2 u_{(0,j)} + u_{(0,j+1)} + u_{(0,j-1)}\right)\right) + \frac{1}{h_{x}} \left(- u_{(0,j)} + u_{(1,j)}\right)\]


    Finally, the boundary condition
\(-u_{x(0,j)}-ik_{(0,j)}u_{(0,j)}=g_{(0,j)}\) gives

    
\[- f_{(0,j)} - k_{(0,j)}^{2} u_{(0,j)} + \frac{2 u_{(0,j)}}{h_{y}^{2}} - \frac{u_{(0,j+1)}}{h_{y}^{2}} - \frac{u_{(0,j-1)}}{h_{y}^{2}} - \frac{2 i}{h_{x}} k_{(0,j)} u_{(0,j)} + \frac{2 u_{(0,j)}}{h_{x}^{2}} - \frac{2 u_{(1,j)}}{h_{x}^{2}}\]

    

    The equation for the point \(x_{(0,j)}\) is
\[  \frac{2 u_{(0,j)}}{h_{y}^{2}}  + \frac{2 u_{(0,j)}}{h_{x}^{2}} - k_{(0,j)}^{2} u_{(0,j)} - \frac{u_{(0,j+1)}}{h_{y}^{2}} - \frac{u_{(0,j-1)}}{h_{y}^{2}}  - \frac{2 u_{(1,j)}}{h_{x}^{2}} - \frac{2 i}{h_{x}} k_{(0,j)} u_{(0,j)}  = 2\frac{g_{(0,j)}}{h_x}+ f_{(0,j)}\]

    On the south boundary, we have (non-corner) points of the form
\((ih_x,0)\), where \(0<i<n_x\). At the point \(x_{(i,0)}\), the
boundary condition is \(−u_{y(i,0)}−ik_{(i,0)}u_{(i,0)}=0\). The
equation for this point is
\[  \frac{2 u_{(i,0)}}{h_{x}^{2}}  + \frac{2 u_{(i,0)}}{h_{y}^{2}} - k_{(i,0)}^{2} u_{(i,0)} - \frac{u_{(i+1,0)}}{h_{x}^{2}} - \frac{u_{(i-1,0)}}{h_{x}^{2}}  - \frac{2 u_{(i,1)}}{h_{y}^{2}} - \frac{2 i}{h_{y}} k_{(i,0)} u_{(i,0)}  = 2\frac{g_{(i,0)}}{h_y}+ f_{(i,0)}\]

    Further, we consider the (non-corner) points on the east boundary, of
the form \((1,jh_y)\), where \(0<j<n_y\). At the point
\(x{(n_x+1,j)}=((n_x+1)h_x,jh_y)=(1,jh_y)\), the boundary condition is
\[u_{x(n_x+1,j)}−ik_{(n_x+1,j)}u_{(n_x+1,j)}=g_{(n_x+1,j)}.\] We
approximate the derivative \(u_{x(n_x+1,j)}\) with backward differences
and obtain
\[u_{x(n_x+1,j)}=h_x^{-1}(u_{(n_x+1,j)}-u_{(n_x,j)})+(1/2)h_xu_{xx(n_x+1,j)}+O(h_x^2)\]
    
\[\frac{h_{x} u_{xx(n_x+1,j)}}{2} + \frac{1}{h_{x}} \left(u_{(n_x+1,j)} - u_{(n_x,j)}\right)\]

    
The second derivative \(u_{xx(n_x+1,j)}\) in the previous expression can
be substituted using the equation \(-u_{xx}-u_{yy}-k^2u=f\) (extended by
continuity to the boundary point (?))


    \[- f_{(n_x+1,j)} - k_{(n_x+1,j)}^{2} u_{(n_x+1,j)} - u_{yy(n_x+1,j)}\]

    

    And the second derivative \(u_{yy(n_x+1\,j)}\) can be approximated using
central differences (with order \(O(h_y^2)\)) to obtain

    \[- f_{(n_x+1,j)} - k_{(n_x+1,j)}^{2} u_{(n_x+1,j)} + \frac{2 u_{(n_x+1,j)}}{h_{y}^{2}} - \frac{u_{(n_x+1,j+1)}}{h_{y}^{2}} - \frac{u_{(n_x+1,j-1)}}{h_{y}^{2}}\]

    We substitute the expression for the second derivative in the expression
of the approximation of the first derivative, to obtain

    \[\frac{h_{x}}{2} \left(- f_{(n_x+1,j)} - k_{(n_x+1,j)}^{2} u_{(n_x+1,j)} - \frac{1}{h_{y}^{2}} \left(- 2 u_{(n_x+1,j)} + u_{(n_x+1,j+1)} + u_{(n_x+1,j-1)}\right)\right) + \frac{1}{h_{x}} \left(u_{(n_x+1,j)} - u_{(n_x,j)}\right)\]

    Finally, the boundary condition
\(u_{x(n_x+1,j)}-ik_{(n_x+1,j)}u_{(n_x+1,j)}=g_{(n_x+1,j)}\) gives

    \[- \frac{f_{(n_x+1,j)} h_{x}}{2} - \frac{h_{x} u_{(n_x+1,j)}}{2} k_{(n_x+1,j)}^{2} + \frac{h_{x} u_{(n_x+1,j)}}{h_{y}^{2}} - \frac{h_{x} u_{(n_x+1,j+1)}}{2 h_{y}^{2}} - \frac{h_{x} u_{(n_x+1,j-1)}}{2 h_{y}^{2}} - i k_{(n_x+1,j)} u_{(n_x+1,j)} + \frac{u_{(n_x+1,j)}}{h_{x}} - \frac{u_{(n_x,j)}}{h_{x}}\]

    
    
    \[- f_{(n_x+1,j)} - k_{(n_x+1,j)}^{2} u_{(n_x+1,j)} + \frac{2 u_{(n_x+1,j)}}{h_{y}^{2}} - \frac{u_{(n_x+1,j+1)}}{h_{y}^{2}} - \frac{u_{(n_x+1,j-1)}}{h_{y}^{2}} - \frac{2 i}{h_{x}} k_{(n_x+1,j)} u_{(n_x+1,j)} + \frac{2 u_{(n_x+1,j)}}{h_{x}^{2}} - \frac{2 u_{(n_x,j)}}{h_{x}^{2}}\]

    

    This leads to the following equation for the point \(x_{(n_x+1,j)}\):

\[ \frac{2 u_{(n_x+1,j)}}{h_{y}^{2}} + \frac{2 u_{(n_x+1,j)}}{h_{x}^{2}} - k_{(n_x+1,j)}^{2} u_{(n_x+1,j)} - \frac{u_{(n_x+1,j+1)}}{h_{y}^{2}} - \frac{u_{(n_x+1,j-1)}}{h_{y}^{2}} - \frac{2 u_{(n_x,j)}}{h_{x}^{2}} - \frac{2 i}{h_{x}} k_{(n_x+1,j)} u_{(n_x+1,j)}   = \frac{2g_{(n_x+1,j)}}{h_x} + f_{(n_x+1,j)} \]

    We finish with the north boundary, where the points have the form
\(x_{(i,n_y+1)}=(ih_x,1)\) with \(0<i<n_x\). Similarly to the previous
cases, the equation for this point is

\[ \frac{2 u_{(i,n_y+1)}}{h_{x}^{2}} + \frac{2 u_{(i,(n_y+1))}}{h_{x}^{2}} - k_{(i,n_y+1)}^{2} u_{(i,n_y+1)} - \frac{u_{(i+1,n_y+1)}}{h_{x}^{2}} - \frac{u_{(i-1,n_y+1)}}{h_{x}^{2}} - \frac{2 u_{(i,n_y)}}{h_{y}^{2}} - \frac{2 i}{h_{y}} k_{(i,n_y+1)} u_{(i,n_y+1)}   = \frac{2g_{(i,n_y+1)}}{h_y} + f_{(i,n_y+1)}\]

    In summary, we have obtained the following equations:

For interior points of the form \(x_{(i,j)}=(ih_x,jh_y)\) where
\(0<i <n_x+1\) and \(0<j < n_y+1\):
\[- \frac{\left(u_{{(i,j+1)}} + u_{{(i,j-1)}}\right)}{h_y^2} -  \frac{\left(u_{{(i+1,j)}} + u_{{(i-1,j)}}\right)}{h_x^2} + \left( 
\frac{2}{h_x^2} u_{{(i,j)}} + \frac{2}{h_y^2} -k_{ij}^{2} \right)u_{{(i,j)}} = f_{ij}\]

For non-corner points on the west boundary, of the form
\(x_{(0,j)}=(0,jh_y)\) where
\(0<j<n_y+1\):\\\[  \frac{2 u_{(0,j)}}{h_{y}^{2}}  + \frac{2 u_{(0,j)}}{h_{x}^{2}} - k_{(0,j)}^{2} u_{(0,j)} - \frac{u_{(0,j+1)}}{h_{y}^{2}} - \frac{u_{(0,j-1)}}{h_{y}^{2}}  - \frac{2 u_{(1,j)}}{h_{x}^{2}} - \frac{2 i}{h_{x}} k_{(0,j)} u_{(0,j)}  = 2\frac{g_{(0,j)}}{h_x}+ f_{(0,j)}\]

For non-corner points on the south boundary, of the form
\(x_{(i,0)}=(ih_x,0)\) where \(0<i<n_x+1\):

\[  \frac{2 u_{(i,0)}}{h_{x}^{2}}  + \frac{2 u_{(i,0)}}{h_{y}^{2}} - k_{(i,0)}^{2} u_{(i,0)} - \frac{u_{(i+1,0)}}{h_{x}^{2}} - \frac{u_{(i-1,0)}}{h_{x}^{2}}  - \frac{2 u_{(i,1)}}{h_{y}^{2}} - \frac{2 i}{h_{y}} k_{(i,0)} u_{(i,0)}  = 2\frac{g_{(i,0)}}{h_y}+ f_{(i,0)}\]

For non-corner points on the east boundary, of the form
\(x_{(n_x,j)}=(1,jh_y)\) where \(0<j<n_y+1\):

\[ \frac{2 u_{(n_x+1,j)}}{h_{y}^{2}} + \frac{2 u_{(n_x+1,j)}}{h_{x}^{2}} - k_{(n_x+1,j)}^{2} u_{(n_x+1,j)} - \frac{u_{(n_x+1,j+1)}}{h_{y}^{2}} - \frac{u_{(n_x+1,j-1)}}{h_{y}^{2}} - \frac{2 u_{(n_x,j)}}{h_{x}^{2}} - \frac{2 i}{h_{x}} k_{(n_x+1,j)} u_{(n_x+1,j)}   = \frac{2g_{(n_x+1,j)}}{h_x} + f_{(n_x+1,j)} \]

For non-corner points on the northern boundary, of the form
\(x_{(i,n_y)}=(ih_x,1)\) where \(0<i<n_x+1\)
\[ \frac{2 u_{(i,n_y+1)}}{h_{x}^{2}} + \frac{2 u_{(i,(n_y+1))}}{h_{x}^{2}} - k_{(i,n_y+1)}^{2} u_{(i,n_y+1)} - \frac{u_{(i+1,n_y+1)}}{h_{x}^{2}} - \frac{u_{(i-1,n_y+1)}}{h_{x}^{2}} - \frac{2 u_{(i,n_y)}}{h_{y}^{2}} - \frac{2 i}{h_{y}} k_{(i,n_y+1)} u_{(i,n_y+1)}   = \frac{2g_{(i,n_y+1)}}{h_y} + f_{(i,n_y+1)}\]

    We continue with the corner points on the boundary of the domain. At the
point \(x_{(0,0)}=(0,0)\), the boundary condition
\(\partial_n u -iku = 0\) in the horizontal direction results in the
equation \[-u_{x(0,0)}-ik_{(0,0)}u_{(0,0)}=g_{(0,0^+)}.\] Similarly, in
the vertical direction the boundary condition is
\[-u_{y(0,0)}-ik_{(0,0)}u_{(0,0)}=g_{(0^+,0)}.\]


    Approximating the derivatives \(u_{x(0,0)}\) and \(u_{y(0,0)}\) by
forward differences leads to
    We multiply the boundary conditions by \(2h_y^{-1}\) and \(2h_x^{-1}\)
respectively


    Summing these two equations we obtain
    
    \[u_{xx(0,0)} + u_{yy(0,0)} - \frac{2 i}{h_{y}} k_{(0,0)} u_{(0,0)} + \frac{2 u_{(0,0)}}{h_{y}^{2}} - \frac{2 u_{(0,1)}}{h_{y}^{2}} - \frac{2 i}{h_{x}} k_{(0,0)} u_{(0,0)} + \frac{2 u_{(0,0)}}{h_{x}^{2}} - \frac{2 u_{(1,0)}}{h_{x}^{2}}\]

    

    Therefore, we have the equation

\[ u_{xx(0,0)} + u_{yy(0,0)} - \frac{2 i}{h_{y}} k_{(0,0)} u_{(0,0)} + \frac{2 u_{(0,0)}}{h_{y}^{2}} - \frac{2 u_{(0,1)}}{h_{y}^{2}} - \frac{2 i}{h_{x}} k_{(0,0)} u_{(0,0)} + \frac{2 u_{(0,0)}}{h_{x}^{2}} - \frac{2 u_{(1,0)}}{h_{x}^{2}} = 2g_{(0^+,0)}h_y^{-1}+ 2g_{(0,0^+)}h_x^{-1}\]

    We substitute in this expression
\(u_{xx(0,0)}+u_{yy(0,0)}=-f_{(0,0)}-k_{(0,0)}^2u_{(0,0)}\)

     
    \[- f_{(0,0)} - k_{(0,0)}^{2} u_{(0,0)} - \frac{2 i}{h_{y}} k_{(0,0)} u_{(0,0)} + \frac{2 u_{(0,0)}}{h_{y}^{2}} - \frac{2 u_{(0,1)}}{h_{y}^{2}} - \frac{2 i}{h_{x}} k_{(0,0)} u_{(0,0)} + \frac{2 u_{(0,0)}}{h_{x}^{2}} - \frac{2 u_{(1,0)}}{h_{x}^{2}}\]

    

    The resulting boundary condition equals

\[ \frac{2 u_{(0,0)}}{h_{y}^{2}} + \frac{2 u_{(0,0)}}{h_{x}^{2}} - k_{(0,0)}^{2} u_{(0,0)}  - \frac{2 u_{(0,1)}}{h_{y}^{2}} - \frac{2 u_{(1,0)}}{h_{x}^{2}} -  \frac{2 i}{h_{y}} k_{(0,0)} u_{(0,0)}- \frac{2 i}{h_{x}} k_{(0,0)} u_{(0,0)} =  f_{(0,0)} + 2g(0^+,0)h_y^{-1}+ 2g(0,0^+)h_x^{-1} \]

    We continue with the southeast corner point \(x_{(n_x+1,0)}=(1,0)\). At
this point the boundary conditions on the vertical and horizontal
directions are \(-u_{y(n_x+1,0)}-ik_{(n_x+1,0)}u_{(n_x+1,0)}=g(1^-,0)\)
and \(u_{x(n_x+1,0)}-ik_{(n_x+1,0)}u_{(n_x+1,0)}=g(1,0^+)\).

The resulting boundary condition is
\[\frac{2u_{(n_x+1,0)}}{h_x^2} + \frac{2 u_{(n_x+1,0)}}{h_y^2} - k^2_{(n_x+1,0)} u_{(n_x+1,0)} -\frac{2u_{(n_x,0)}}{h_x^2} - \frac{2u_{(n_x+1,1)}}{h_y^2}-  \frac{2i}{h_y} k_{(n_x+1,0)} u_{(n_x+1,0)} - \frac{2i}{h_x} k_{(n_x+1,0)} u_{(n_x+1,0)}=f_{(n_x+1,0)} + 2g(1^{-},0)h_y^{-1}+2g(1,0^+)h_x^{-1}\]

At the northwest corner point \(x_{(0,n_y+1)}=(0,1)\) the equation is
\[\frac{2u_{(0,n_y+1)}}{h_x^2} + \frac{2 u_{(0,n_y+1)}}{h_y^2} - k^2_{(0,n_y+1)} u_{(0,n_y+1)} -\frac{2u_{(1,n_y+1)}}{h_x^2} - \frac{2u_{(0,n_y)}}{h_y^2}-  \frac{2i}{h_y}  k_{(0,n_y+1)}  u_{(0,n_y+1)} - \frac{2i}{h_x} k_{(0,n_y+1)}u_{(0,n_y+1)}=f_{(0,n_y+1)} + 2g(0^{+},1)h_y^{-1}+2g(0,1^-)h_x^{-1}\]

    At the northeast corner point \(x_{(n_x+1,n_y+1)}=(1,1)\) the equation
is

\[\frac{2u_{(n_x+1,n_y+1)}}{h_x^2} + \frac{2 u_{(n_x+1,n_y+1)}}{h_y^2} - k^2_{(n_x+1,n_y+1)} u_{(n_x+1,n_y+1)} -\frac{2u_{(n_x,n_y+1)}}{h_x^2} - \frac{2u_{(n_x+1,n_y)}}{h_y^2}-  \frac{2i}{h_y} u_{(n_x+1,n_y+1)} - \frac{2i}{h_x}u_{(n_x+1,n_y+1)}=f_{(n_x+1,n_y+1)} + 2g(1^-,1)h_y^{-1}+2g(1,1^-)h_x^{-1}\]

    END OF REVISED VERSION
    The complete set of equations is the following:

For interior points of the form \(x_{(i,j)}=(ih_x,jh_y)\) where
\(1<i<n_x\) and \(1<j<n_y\):
\[- h_{x}^{2} \left(u_{{(i,j+1)}} + u_{{(i,j-1)}}\right) - h_{y}^{2} \left(u_{{(i+1,j)}} + u_{{(i-1,j)}}\right) + u_{{(i,j)}} \left(- h_{x}^{2} h_{y}^{2} k_{ij}^{2} + 2 h_{x}^{2} + 2 h_{y}^{2}\right)= f_{ij}h_{x}^{2}h_{y}^{2}\]

For non-corner points on the west boundary, of the form
\(x_{(0,j)}=(0,jh_y)\) where
\(1<j<n_y\):\\\[ - 2 h_{y}^{2} u_{{(1,j)}} - h_{x}^{2} u_{{(0,j+1)}} - h_{x}^{2} u_{{(0,j-1)}}  + u_{{(0,j)}} \left(-h_{x}^{2} h_{y}^{2} k_{{(0,j)}}^{2}  - 2 i h_{x} h_{y}^{2} k_{{(0,j)}} + 2 h_{x}^{2}+ 2 h_{y}^{2}\right)= f_{{(0,j)}} h_{x}^{2} h_{y}^{2} \]

For non-corner points on the south boundary, of the form
\(x_{(i,0)}=(ih_x,0)\) where
\(1<i<n_x\):\\\[ -2 h_{x}^{2} u_{{(i,1)}} - h_{y}^{2} u_{{(0,j+1)}} - h_{y}^{2} u_{{(0,j-1)}} + u_{{(i,0)}} \left(-h_{x}^{2} h_{y}^{2} k_{{(i,0)}}^{2} - 2 i h_{x}^{2} h_{y} k_{{(i,0)}} + 2 h_{x}^{2} + 2 h_{y}^{2}\right)= f_{{(i,0)}} h_{x}^{2} h_{y}^{2} \]

For non-corner points on the east boundary, of the form
\(x_{(n_x,j)}=(1,jh_y)\) where
\(1<j<n_y\):\\\[ - 2 h_{y}^{2} u_{{(n x-1,j)}} - h_{x}^{2} u_{{(n x,j+1)}} - h_{x}^{2} u_{{(n x,j-1)}}  + u_{{(n x,j)}} \left(- h_{x}^{2} h_{y}^{2} k_{{(n x,j)}}^{2}  - 2 i h_{x} h_{y}^{2} k_{{(n x,j)}}+ + 2 h_{x}^{2} + 2 h_{y}^{2}\right)=f_{{(n x,j)}} h_{x}^{2} h_{y}^{2}\]

For non-corner points on the northern boundary, of the form
\(x_{(i,n_y)}=(ih_x,1)\) where \(1<i<n_x\)
\[- 2 h_{x}^{2} u_{{(i,ny-1)}} - h_{y}^{2} u_{{(i+1,n y)}} - h_{y}^{2} u_{{(i-1,ny)}} + u_{{(i,n y)}} \left(- h_{x}^{2} h_{y}^{2} k_{{(i,n y)}}^{2} - 2 i h_{x}^{2} h_{y} k_{{(i,n y)}} + 2 h_{x}^{2} + 2 h_{y}^{2}\right)= f_{{(i,n y)}} h_{x}^{2} h_{y}^{2} \]

For the corner point \(x_{(0,0)}=(0,0)\)
\[- h_{x}^{2} u_{{(0,1)}} - h_{y}^{2} u_{{(1,0)}} + u_{{(0,0)}} \left(- \frac{h_{x}^{2} k_{{(0,0)}}}{2} h_{y}^{2} - i h_{x}^{2} h_{y} k_{{(0,0)}} + h_{x}^{2} - i h_{x} h_{y}^{2} k_{{(0,0)}} + h_{y}^{2}\right)=\frac{f_{(0,0)}h_x^2h_y^2}{2}\]

For the corner point \(x_{(n,0)}=(1,0)\)
\[- h_{x}^{2} u_{{(n,1)}} - h_{y}^{2} u_{{(n-1,0)}} + u_{{(n,0)}} \left(- \frac{h_{x}^{2} h_{y}^{2}}{2} k_{{(n,0)}}^{2} - i h_{x}^{2} h_{y} k_{{(n,0)}} + h_{x}^{2} - i h_{x} h_{y}^{2} k_{{(n,0)}} + h_{y}^{2}\right)= \frac{f_{{(n,0)}} h_{x}^{2}h_{y}^2}{2}\]

For the corner point \(x_{(0,n)}=(0,1)\)
\[- h_{x}^{2} u_{{(0,n-1)}} - h_{y}^{2} u_{{(1,n)}} + u_{{(0,n)}} \left(\frac{h_{x}^{2} h_{y}^{2}}{2} k_{{(0,n)}}^{2} - i h_{x}^{2} h_{y} k_{{(0,n)}} + h_{x}^{2} - i h_{x} h_{y}^{2} k_{{(0,n)}} + h_{y}^{2}\right)=  \frac{f_{{(0,n)}} h_{x}^{2}h_y^2}{2}\]

For the corner point \(x_{(n,n)}=(1,1)\)
\[-h_{x}^{2} u_{{(n,n-1)}} - h_{y}^{2} u_{{(n-1,n)}} + u_{{(n,n)}} \left(- \frac{h_{x}^{2} h_{y}^{2}}{2} k_{{(n,n)}}^{2} - i h_{x}^{2} h_{y} k_{{(n,n)}} + h_{x}^{2} - i h_{x} h_{y}^{2} k_{{(n,n)}} + h_{y}^{2}\right)= \frac{f_{{(n,n)}} h_{x}^{2} h_{y}^{2}}{2} \]


    % Add a bibliography block to the postdoc
    
    
    
    \end{document}
